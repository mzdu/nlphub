[grocery]
We present an approach to social grocery shopping based on customers trading information about item prices and quantities in order for the customers to find the lowest prices for the goods they purchase and the most convenient plan for buying them. Because collecting and reporting prices is tedious, agents representing customers are needed to make this approach practical. Agents also have the potential to learn which other agents can be trusted. We use a realistic shop-ping list based on the U.S. Consumer Price Index in order to guarantee the realism of our results. By visiting actual grocery markets and comparing prices, we have discovered that the total cost of a list of groceries can vary by 13\%. We have also discovered that by shopping optimally, that is, buying each item from the cheapest store, the result can be a savings of 16\% over shopping at the store with the lowest total cost. To shop at minimum cost requires customers' agents to report prices to each other. If they do, each customer is likely to achieve at least a 10\% savings. However, what if the reported prices are inaccurate? Would customers be worse off than if they just shopped randomly? We investigated the robustness of our multiagent shopping system in the presence of errors in reported prices. From this, we determine the potential savings an average customer might obtain.

[healthcare]
Societal information systems are intended to assist the members of a society in dealing with the complexities of their interactions with each other, especially regarding the resources they share. Because the members are distributed and autonomous, we believe that software agents, having these same characteristics, are a natural basis for representing the members and their interests in a societal information system.  This paper describes a simulation of an agent-based societal information system for healthcare. Our design methodology is based on agent-oriented modeling, which we apply to analyze and design the system and its simulation. We execute the simulation to investigate four different strategies for assisting a person in choosing a physician, combined with three waiting strategies in three common social network models. The results show that the societal information system can decrease the number of annual sick days per person by 0.42-1.84 days compared with choosing a physician randomly.

[personality]
Because of the large number of agents and robots beginning to affect everyday life of humans, it is important to understand how humans would treat agents in a mixed human-agent society. In this paper, we are trying to find answers to two questions: whether humans possess different attitudes towards other humans and agents, and whether the personality type of a human influences his/her decisions and how. To investigate these problems, first we use the Keirsey Temperament Sorter-II (KTS-II) to discover the personality types of our human participants. Then each participant plays the "Who Gets More Cake?" game three times, with a simulated human and an agent as opponents. The experimental results are shown in two aspects: the tendency aspect and the consistency aspect. It is shown that humans treat other humans and agents differently and humans with different KTS-II temperaments behave differently on the above two aspects. It is very possible that the Thinking--Feeling dichotomy of Myers-Briggs Type Indicator (MBTI) and the tendency results are not independent. Also, there is a correlation between the Extraversion--Introversion dichotomy and the consistency results.