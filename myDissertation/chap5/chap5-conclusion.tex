% ********** Chapter 6 **********
\chapter{Conclusion}
\label{ch5-conclusion}

Agents are involved in humans' everyday life and thus humans with the help of agent technologies form different sociotechnical systems under various situations. The reason why humans need technical help is the increasing complexity of ongoing problems in the society. For the complex problems that are beyond the ability of individuals, it is promising to take advantage of advanced technologies to cope with them. For technical help, we consider two aspects, which are how to represent and use the interactions of agents acting on behalf of humans, and how a person interacts with the system or the agent. In this dissertation, we investigated these two aspects by studying three cases using the case study approach.

The first case is a multiagent shopping system that could give a customer suggestions on where to shop and what to shop. We proposed an approach, tested it with simulated price data and real price data collected from nearby stores, and compared it with three other common approaches. It is shown that at least 22\% savings could be made generally with simulated data. With the real data, a customer could save 6\% more using our approach. We also proved robustness of our approach under the situation of deceptive stores with simulated data and wrongly reported price with real data.  
 
The second case is a multiagent healthcare system in which agents representing patients could exchange information about physicians and recommend the most proper physician to a patient. We considered three kinds of networks: random network, scale-free network, and small-world network to represent the social relationship of people. We assume that if two people know each other, their agents know each other too. The agent of a patient asks the agents of the patient's friends for information about physicians they know. Depending on different choosing-a-physician strategies and waiting policies, the patient may get different recommendations of a physician. As for the number of annual sick days per person in different situations, it is shown that 0.4-1.8 days could be reduced using the sociotechnical system compare to that of choosing a physician randomly. 

In the third case, we are trying to investigate a factor, personality to be particular, that influences human-agent interaction. To achieve this purpose, we test human subjects' personality and have them play the "Who Gets More Cake?" game. The game is designed in such a way that at the end of the game each subject is asked a question that indicates his inclination towards humans or agents. The result shows that humans treat other humans and agents differently and humans with different personalities behave differently. However, fair is more important than personality types, which means the subjects would try to treat humans and agents differently, but in the same extent of difference regardless of their personalities.     

Agents are involved in the sociotechnical systems in all three cases. In the grocery shopping case, agents work together indirectly through the central manager of the online system. If we could scan the items and the information related to the items could be taken and uploaded by an agent possibly installed on our cellphone automatically to the online system, people could save a lot of money by sharing the information they know. In the healthcare case, agents communicate and collect more information than that could be collected by only one person and utilize integrated information to give suggestions. In the human-agent interaction case, humans' reaction to other humans and agents are observed and the effect of personality as a factor is investigated. Through the three cases, it is shown that agents, which could be considered as a piece of autonomous intelligent software, could provide suggestions to humans by integrating information that a human could not or does not willing to handle. Along with the rapid development of technologies and thus huge amount of information, the need of agents in different areas of human life is growing bigger and bigger. Thus, we foresee that agent technologies will be widely used in the future than it is today.

However, there are limitations in our case studies. For the shopping system, we need some way to get the item information automatically to the online system, which might be difficult because stores may not want to provide the convenience of getting item information they have already stored in their system by simply letting customers scan the barcode.

For the healthcare system, real life is much more complex than our simulation. First, we simplify the question by making the patients and physicians homogeneous, that is, each person behaves similarly to the other persons of the same kind. For example, the ratings of a physician from different patients won't vary too much; each physician could take care of the same number of patients at most every day; the number of patients distributed evenly throughout the year, etc. We assume that patients all adopt the same choosing-a-physician strategy and waiting policy in a particular simulation, which is not true in the real world since everyone might have different preferences over these choices and each person may have different preferences depending on the severity of his disease. Second, we didn't consider unforeseen situations or the variables. For example, a patient may change his/her idea in the middle of the treatment, such as switching to another physician if he/she is not satisfied with the current physician.  

In the human-agent interaction case, there might be multiple factors that influence humans' decisions in the game. We didn't consider the interactions of different factors and actually it's even impossible to list all the factors. Also, the subjects in our experiment are students around 20s. If we want to generalize the conclusion, we need to collect data from more subjects of various ages, educational background and so on. In this way, we could get more useful and accurate conclusion. Moreover, if we have large amount of subjects of different background play different games, we might find some pattern that could guide us and learn more about the effects of personality or other factors on humans' attitude towards other humans and agents.

Based on our experience, here are some recommendations for designing sociotechnical systems:
\begin{itemize}
\item[-]Model the problem as realistically as possible. When designing a sociotechnical system for research purposes, simplifications are usually made because it is difficult to make it just like in a complex real life situation. However, it is necessary to model the problem in question complex enough to make sure it doesn't lack of any critical factors that might influence the conclusion. 
\item[-]Carefully design the rules of the interaction among agents, and the interaction between a human and an agent. Different systems may require different rules of interaction, so it is important to define the norms and have all the agents follow the norms.   
\item[-]Consider possible variables in the system and perform a robustness analysis. In most systems, there are variables that may change at some point or wrong data input depending on the designed system. These uncertainties should be considered and taken care of.
\end{itemize}

To summarize, agent technologies can have important effect on aiding or facilitating humans' everyday life in various kinds of sociotechnical systems. To better accomplish this purpose, we need to understand how the agents could help in sociotechnical systems, which motivates this dissertation. There are additioanl studies that could be done to improve the work in this dissertation. For the shopping suggestion system, a mobile app could be developed and used to upload the information to the central manager; real shopping lists of different types of families could be used to further test the system. For the healthcare system, a more complex model is needed to overcome the shortcomings mentioned in the limitations part. For the human-agent interaction case, more subjects of different background could be very helpful to draw a more accurate conclusion. Exploring different possibilities of improving these case studies could help us to understand aspects of implementing sociotechnical systems more in depth, thus providing useful tools for assisting humans with complex problems that are hard to cope with.             