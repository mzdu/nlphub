% ********** Chapter 1 **********
\chapter{Introduction}
\label{ch0}

As the amount of communication among people increases rapidly and the technology grows fast, more complex problems occurs and humans don't have enough time, energy, or ability to tackle with all these problems. Our work involves one complex problem -- scarce resource allocation problem. To help humans with these problems in sociotechnical systems, we took advantage of multiagent systems in which autonomous agents represent humans' interests and make decisions for humans. 

In implementing such systems, we need to handle two technical issues. First, how the agents represent humans and how the agents interact with each other in the systems. Second, how we represent the interaction between humans and agents. Our work contains three study cases, while the first two cases, shopping route optimization and health care provider selection through recommendations investigate the first issue. The third case is about human-agent interaction while playing a game, which explores the second technical issue. 

This introduction continues with more detailed explanation for the purpose of study in section \ref{ch0:PurposeOfStu} and a description of complex problems and possible solutions in section \ref{ch0:problemsAndSol}. To find solutions for these problems, we need some technical help from the artificial intelligence domain. In section \ref{ch0:MotivatingExa}, we explain what motivates our study cases in this work.

\section{Purpose of Study}
\label{ch0:PurposeOfStu}

Because human societies are complex and humans rely on resources that are scarce, we encounter complex resource allocation problems. Scarce resources might be different under different circumstances, such as food, energy, medical care, clean water, etc. To solve these problems, human interact with each other, and the interaction or information flow among them forms different kinds of information systems. Together with the technical aspects that participate in the systems to help, the information systems are actually sociotechnical systems, as we will define later. There are various kinds of sociotechnical systems serving different purposes. Take a simple situation as an example, a patient is trying to find a doctor who could take care of him. He may have a goal to spend money as less as possible, or to cure him as soon as possible, but he has no idea which doctor fits his purpose best. Thus he turns to his friends, his friends' friends if necessary, for recommendations, incorporate these information into consideration, and make a decision with the help of agents. In this example, the persons involved, the agents and the information flow form a sociotechnical system and the purpose is to find a doctor for the patient. There are more complex information systems like the ones in section \ref{ch0:problemsAndSol}. 

As the scale of a system grows, more complex problems, some with global interactions which lead to global consequences, emerge. In many cases it is hard to find solutions to the problems or perform experiments on real systems due to various reasons, such as difficulty of synchronization, long time span of doing the experiments, or extreme geological conditions.

Because of complexity of the systems, the huge amount of information flow and other factors that make the problems hard to solve, humans need computational help in designing, implementing, and evaluating the systems. With the technical help of simulated systems, the cost of experiments is reduced and the purpose of study is fulfilled. Several questions need to be considered while designing a simulated sociotechnical system. For instance, how big should the system be and what entities are involved? What kind of information should be kept an eye on? What consequences to expect and what goals to achieve? We'll see more analysis in system design in section \ref{ch0:problemsAndSol}. 

Humans involved in complex problems usually only have access to partial information and they try to achieve a common goal while pursuing their own interests. This characteristic is consistent with the feature of distributed systems where agents with partial information are used to assist humans to make decisions. Such systems with multiple agents are called MultiAgent Systems (MAS). Agents are an autonomous software entities that can act on the behalf of his principle, sometimes a human in a sociotechnical system, based on his knowledge and judgment. This natural characteristic makes an agent a good representative of a human. Also, the amount of information flow in a sociotechnical system could be potentially enormous, which is beyond the processing ability of humans brain, thus it is better to have an autonomous agent gather information, communicate with other agents and make decisions on behalf of a person. A multiagent system is a system/society that gathers multiple agents who interact with each other. Information is exchanged among the agents who have goals based on their principles' interests. Nowadays agent technology is used everywhere, ranging from industry such as fault detection, energy distribution, to everyday life, such as web services, security patrols. First two case studies in this dissertation use multiagent systems to implement a grocery shopping scenario and a health care system.

Due to the popularity of the agents existed in our society it is inevitable that human-agent mixed societies emerge. In such societies, humans and agents exchange information and work together to achieve a particular goal, compete with each other, or have more complex relationships. Examples of working together include teaching children languages or mathematics using emotional agents. An emotional agent is an agent with emotions which are expressed by expressions of its animated face on the screen, words programmed in it, and so on. If a child answers a question correctly or performs well, the agent smiles or does other positive expressions and actions. An example of competition is that humans and agents take part in an auction and bid for some goods on the Internet. 

It is important to understand how humans and agents interact in various human-agent mixed societies in different aspects. For example, will humans have the same performance in the mixed societies as previous while there's no agent involved? What factors influence humans' decisions/attitude towards agents? Do humans' personalities play a part in their decisions and how? Many researchers studied the first two questions but less studied the third question. The third part of this dissertation is trying to get an insight into the questions about relationship between personalities and decisions. Conclusions to these questions could be used in many ways. For example, we could predict the performance of humans in a game knowing their personalities, or assign an agent with the "proper" personality to accompany a human, etc.     

\section{Complex Problems and Possible Solutions}
\label{ch0:problemsAndSol}

The problems human encounter everyday range from very personal, such as what to eat for breakfast, to very influential, such as what the best plan is for a company. Nowadays problems become more and more complicated, considering the following three factors:
\begin{itemize}
\item[-]size: since the communication of people and exchange of information are very frequent today due to the development of new technologies and market needs, it is very possible that problems encountered have larger size than ever before. For example, people like to take digital pictures and put it on the Internet, and with the increasing size of digital photos today, it takes a lot of space to store these photos and more time to find specific photos. Another example is integrating several databases of huge amount of data. Because the databases are huge and there are complex relationships among them, any operation should be considered or evaluated before they are actually performed. The size of a problem matters because it could motivate new technologies which deal with new challenges brought by the size.
\item[-]intersection: a problem may involve different areas and intersect or overlap with other problem domains. For example, consider the problem of arranging the routes of goods transportation of a delivery company everyday. First a couple of key time points should be considered, such as the arrival time of goods to the company. Other things to be considered include available transportation vehicles and human labors, weather, and so on. This problem involves human resources, scheduling, in addition with the help of weather forecasting, and some other areas. For complex problems, it is inevitable that they involve different areas and it is beyond the capacity of just humans, which is why we need the help of technology.     
\item[-]consequences: due to the above two factors and globalization, some problems today have more influential consequences than before. For example, an erroneous operation on databases of a large electricity company may lead to failure of several power plants, causing residents of an area short of electricity. Another example is global warming, which caused by multiple reasons. Possible reasons include increasing size of people and cars therefore more carbon dioxide, decreasing area of forests, polluted air and seas, and so on, which are all interrelated that the problem couldn't be solved with the effort of only a portion of people. Some events, such as nuclear disaster, happen on one location of the world, but continuously have global consequences, such as the release of radioactive materials.              
\end{itemize}

Because of these factors, some problems are so complex that technical help designed to deal with the complexity of these problems is needed. There are two parts or aspects of the technical help: 
\begin{itemize}
\item[-]How a technical system represents each person and their interactions with each other and the problem domain. For example, we need to decide how the agents communicates in a multiagent system under a particular situation.
\item[-]How a person interacts with the system. For example, we could have a person specify his preferences by selecting some options on a web page.
\end{itemize}
Since multiagent systems represent the feature of complex problems well in the sense that information centralization isn't a must in the systems and that autonomous agents could represent persons well, multiagent systems are used in this work.

\section{Research Methodology}
\label{ch0:MotivatingExa}
For the various complex problems existed in the sociotechnical systems everywhere nowadays, humans don't have time or interests to work with other humans on these problems, so humans need agents to represent them which could relieve them from tasks or pressure. Therefore, we turned to multiagent systems for technical help. 

As for research methodology, which should depend on the research questions, we uses case study approach to investigate the aspects of implementing a sociotechnical system in depth. As Yin \cite{yin2009} said, case study is "an empirical inquiry about a contemporary phenomenon (e.g., a "case"), set within its real-world context-especially when the boundaries between phenomenon and context are not clearly evident." Case study could be used if the research addresses a descriptive question or an explanatory question, such as "What is happening?", or "Why or how is it happening?" \cite{shavelson2003}, which is exactly what we need. 

In this work, we studied three cases. First, can you imagine an agent would help you to list the goods that you want to buy just by scanning the barcode of your goods that's running out or taking pictures of them, calculate the optimal route based on your location and your preferences such as which stores you usually visit, and provide suggestions? In the first case study, we try to find the optimal route for a customer who wants to do shopping. First we rely on a multiagent system to publish and retrieve information related to the items sold in store, such as price and quality. Then an agent representing the customer will provide a solution for the customer according to the shopping list by giving suggestions of which stores to visit. We used simulated data and real-world data to test our approach and then evaluated the robustness of the system.

Second, have you ever troubled by the question of which physician or doctor to visit when you are ill? How would you know whom is good for you, especially if you don't have experience with any of them? Of course you could search online, but the information there may be misleading and outdated. In our healthcare system, an agent representing you could interact with the agents of your friends, or even your friends' friends, to acquire information, integrate them, and make a suggestion based on your preference, such as saving money, or heal fast. Friends' agents have their choices of whether to respond to the patient's agent or not. In this case, we investigate the interaction among agents.  

Third, do you like or fear to interact with agents? Have you wondered what factors influence your emotion or affection towards agents? These questions are encountered inevitably while designing a multiagent system. In our last case, we studied the effect of a particular factor - personality - on the decisions humans made while interacting with agents and other humans in a mixed human-agent society. Human subjects were guided to play a variant of cake-cutting game and then asked a question of how they would like to divide the leftover cake between the simulated human and the agent participated in the game. So the questions are, would personality play a part in humans' decisions and is there any pattern for the answers to the question?

The three cases come from different domains and it seems that they are unrelated, but actually not. In the shopping scenario, agents contributed data to a central server and receive data from a central server, so the agents interacted indirectly, but a customer's shopping agent and other persons' shopping agents might not talk to each other. In the healthcare case, agents work with or interact directly with other agents on the problem. In the human-agent interaction case, we investigate how a person would interact with his agent, and how an agent would interact with a person and with other agents. Therefore, the first two cases studied the first aspect of the technical help and the third case investigated the second aspect. The three cases contribute to what we need for implementing a simulated sociotechnical system. 
  
% ********** End of chapter **********
